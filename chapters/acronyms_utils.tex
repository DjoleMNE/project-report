%!TEX root = ../report.tex
\usepackage[toc,nonumberlist,acronym, nomain]{glossaries}
\usepackage{calc}
\usepackage{booktabs}
\usepackage{tabularx}% (for comparison)
\makeglossaries

\newlength\maxlength
\newlength\thislength



\newglossarystyle{mystyle}
{%
	\renewenvironment{theglossary}%
	{% start of glossary
		% Find maximum width of the first column:
		\setlength{\maxlength}{0pt}%
		\forglsentries[\currentglossary]{\thislabel}%
		{%
			\settowidth{\thislength}{\glsentryshort{\thislabel}}%
			\ifdim\thislength>\maxlength
			\setlength{\maxlength}{\thislength}%
			\fi
		}%
		% Now calculate the width of the second column:
		\settowidth{\thislength}{\hspace{10.5em}=\hspace{1em}}%
		\setlength{\glsdescwidth}{\linewidth-\maxlength-\thislength-2\tabcolsep}%
		% Start the tabular environment
		\begin{tabular}{l@{\hspace{5.2em}=\hspace{2.0em}}p{\glsdescwidth}}
			\toprule
			\multicolumn{1}{l}{\textbf{Abbreviation}} &
			\multicolumn{1}{@{}l}{\textbf{Meaning}}\\%
			\midrule
		}%
		{% end of glossary
			\bottomrule
		\end{tabular}%
	}%

% Header has been incorporated into \begin{theglossary}
\renewcommand*{\glossaryheader}{}%
% Don't do anything between letter groups
\renewcommand*{\glsgroupheading}[1]{}%
\renewcommand*{\glsgroupskip}{}%
% Set display for each the acronym entry
\renewcommand{\glossentry}[2]{%
	\glstarget{##1}{\glsentryshort{##1}}% short form
	&
	\glsentrylong{##1}% long form
	\\% end of row
}%
% No sub-entries, so \subglossentry doesn't need redefining
}